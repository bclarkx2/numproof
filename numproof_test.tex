% Class

\documentclass{homework}


% Info

\student{Brian Clark}
\course{EECS 391}
\assignment{HWX}
\duedate{Never}


% Packages

\usepackage{amsmath}
\usepackage{nccmath}
\usepackage{graphicx}
\usepackage{enumitem}
\usepackage{textcomp}
\usepackage{lipsum}
\usepackage{numproof}
\usepackage{amssymb} 	

% Settings

\setlength{\parindent}{0pt}
\setlength{\parskip}{1em}
\graphicspath{ {img/} }


% Commands

\newcommand{\bol}[1]{\textbf{#1}}
\newcommand{\pof}[1]{\textbf{P}(#1)}


% Document

\begin{document}
\maketitle


\begin{numproof}{Proof of Brian's Theorem}
	\item Hello, this is line number 1.\comline{Modus Ponens}{3-4}
	\item This is line number 2\comline{Modus Tollens}{5}
	\item line 3, has comment without line number \com{Normal comment}
	\begin{subproof}{3}
		\item this is a subproof, first item
		\item subproof, line 2 \com{Subproofs can have comments}
		\item subproof, line 3 \comline{Even lined comments}{2-6}
	\end{subproof}
	\item Return to normal lines
	\item This line has a note! \note{This could be quite a lengthy explanation. \lipsum[1]}
	\item This line also has a note, with a different symbol. \note{Hello!}
	\item Here's a line with a subproof
		\begin{subproof}{7}
			\item Subproof 1
			\item Subproof 2
			\item Subproof 3 - This line requires another level
				\begin{subproof}{c}[\arabic*]
					\item l1
					\item l2
				\end{subproof}
			\item Subproof 4
		\end{subproof}
	\item Back to regular lines
	\item Hello!
	\item Wasting space!
	\item This line is really long, and also has a comment! \lipsum[4] \comline{White Path Theorem}{2}
	\item Another normal line!
	\item Theorem = proved
\end{numproof}


\newpage
\begin{numproof}{Proof Number 2}
	\item Notes in this proof start over with the symbols. \note{different symbol!}
\end{numproof}













\end{document}